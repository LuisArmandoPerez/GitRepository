\documentclass[a4paper,11pt]{article}
\usepackage[T1]{fontenc}
\usepackage[utf8]{inputenc}
\usepackage{lmodern}

\title{}
\author{}

\begin{document}

Demostraciones Trigonométricas.\\

(Uso algunas igualdades que doy por hecho que se conocen.)\\

Demuestra qué:\\

$sen(2\alpha) = \frac{2tan(\alpha)}{1+tan^2 (\alpha)}$\\

Solución:\\

Sabemos qué:\\

$sen(2\alpha)=sen(\alpha+\alpha)=sen(\alpha)cos(\alpha)+sen(\alpha)cos(\alpha)=2cos(\alpha)sen(\alpha)$\\

$\Rightarrow sen(2\alpha)=2cos(\alpha)sen(\alpha)$\\

Multiplicamos la relación anterior por 1, donde 1 va a ser de la forma:\\

$\frac{cos(\alpha)}{cos(\alpha)}=1$\\

Entonces:\\

$sen(2\alpha)=2cos(\alpha)sen(\alpha)\frac{cos(\alpha)}{cos(\alpha)}$\\

$2cos(\alpha)sen(\alpha)\frac{cos(\alpha)}{cos(\alpha)}=2cos^2(\alpha)\frac{sen(\alpha)}{cos(\alpha)}$\\

Ahora, sabemos qué\\

$\frac{sen(\alpha)}{cos(\alpha)}= tan(\alpha)$\\

y qué\\

$cos(\alpha)=\frac{1}{sec(\alpha)}$ $\Rightarrow$ $cos^2(\alpha)=\frac{1}{sec^2(\alpha)}$\\

Pero también sabemos qué\\

$sec^2(\alpha)=1+tan^2(\alpha)$\\

$\Rightarrow cos^2(\alpha)=\frac{1}{1+tan^2(\alpha)}$\\

Entonces:\\

$2cos^2(\alpha)\frac{sen(\alpha)}{cos(\alpha)}=2\frac{1}{sec^2(\alpha)}tan(\alpha)$\\

$\Rightarrow 2\frac{1}{sec^2(\alpha)}tan(\alpha)=2\frac{1}{1+tan^2(\alpha)}tan(\alpha)$\\

Por lo tanto:\\

$sen(2\alpha) = \frac{2tan(\alpha)}{1+tan^2 (\alpha)}$\\

Qué era lo qué queríamos demostrar.\\











\end{document}
